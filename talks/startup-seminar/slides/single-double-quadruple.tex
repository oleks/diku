\begin{frame}

\frametitle{Half, Single, Double, Quadruple Precision! (1/3)}

\vspace{\fill}

\begin{center}

\begin{tabular}{r|l|l|l}
& Total & Exponent & Significand \\ \hline
Half & \texttt{16} & \texttt{5} & \texttt{10} \\
Single & \texttt{32} & \texttt{8} & \texttt{23} \\
Double & \texttt{64} & \texttt{11} & \texttt{52} \\
Quadruple & \texttt{128} & \texttt{15} & \texttt{112} \\ \hline
\end{tabular}

\end{center}

\vspace{\fill}

\begin{center}

You've probably heard of single and double precision.

\vspace{\fill}

What is the difference in terms of the trade-offs above?

Higher means better, right? Right?

\end{center}

\vspace{\fill}

\end{frame}


\begin{frame}

\frametitle{Half, Single, Double, Quadruple Precision! (2/3)}

\begin{columns}[t]

\begin{column}{0.5\textwidth}

\begin{itemize}

\item \only<1>{Performance}\only<2->{{\color{lightblue}Performance}}

\item Accuracy

\item \only<1-3>{Range}\only<4->{{\color{lightblue}Range}}

\end{itemize}

\end{column}

\begin{column}{0.5\textwidth}

\begin{itemize}

\item \only<1-2>{Ease of use}\only<3->{{\color{lightblue}Ease of use}}

\item \only<1>{Memory use}\only<2->{{\color{lightblue}Memory use}}

\item \only<1>{Power use}\only<2->{{\color{lightblue}Power use}}

\end{itemize}

\end{column}

\end{columns}

\vspace{\fill}

\begin{itemize}

\item<2-> Single and double are typically supported hardware.

\begin{itemize}

\item Performance is good.

\item Higher precision requires more resources.

\end{itemize}

\item<3-> Programmer can typically choose the precision.

\begin{itemize}

\item Interface not always clear-cut, but usually doable.

\end{itemize}

\item<4-> Higher precision means higher range of exponent.

\end{itemize}

\vspace{\fill}

\end{frame}


\begin{frame}

\frametitle{Half, Single, Double, Quadruple Precision! (3/3)}

\vspace{\fill}

\begin{itemize}

\item Higher precision does not guarantee better
accuracy\cite{cuyt-et-al-2001}.

\item Identical output in higher precision is unreliable\cite{cuyt-et-al-2001}.

\item Instability can happen without cancellation\cite{higham-2002}.

\end{itemize}

\vspace{\fill}

\footnotesize

\begin{thebibliography}{9}

\bibitem{cuyt-et-al-2001} A. Cuyt, B. Verdonk, S. Becuwe, and P. Kuterna.
\emph{A Remarkable Example of Catastrophic Cancellation Unraveled}. Computing
66(3): 309--320, 2001.

\bibitem{higham-2002} Nicholas J. Higham. \emph{Accuracy and Stability of
Numerical Algorithms}. Second Edition. SIAM, 2002.

\end{thebibliography}

\end{frame}
