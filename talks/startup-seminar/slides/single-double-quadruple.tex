\begin{frame}

\frametitle{Half, Single, Double, Quadruple Precision! (1/2)}

\vspace{\fill}

\begin{center}

You've probably heard of single and double precision.

\vspace{\fill}

What is the difference in terms of the trade-offs above?

Higher means better, right? Right?

\end{center}

\vspace{\fill}

\end{frame}

\begin{frame}

\frametitle{Half, Single, Double, Quadruple Precision! (2/2)}

\begin{columns}[t]

\begin{column}{0.5\textwidth}

\begin{itemize}

\item \only<1>{Performance}\only<2->{{\color{lightblue}Performance}}

\item Accuracy

\item \only<1-3>{Range}\only<4->{{\color{lightblue}Range}}

\end{itemize}

\end{column}

\begin{column}{0.5\textwidth}

\begin{itemize}

\item \only<1-2>{Ease of use}\only<3->{{\color{lightblue}Ease of use}}

\item \only<1>{Memory use}\only<2->{{\color{lightblue}Memory use}}

\item \only<1>{Power use}\only<2->{{\color{lightblue}Power use}}

\end{itemize}

\end{column}

\end{columns}

\vspace{\fill}

\begin{itemize}

\item<2-> Single and double are typically supported hardware.

\begin{itemize}

\item Performance is good.

\item Higher precision requires more resources.

\end{itemize}

\item<3-> Programmer can typically choose the precision.

\begin{itemize}

\item Interface not always clear-cut, but usually doable.

\end{itemize}

\item<4-> Higher precision means higher range of exponent.

\begin{itemize}

\item Single: 8-bit exponent; Double: 11-bit exponent.

\end{itemize}

\end{itemize}

\vspace{\fill}

\end{frame}
