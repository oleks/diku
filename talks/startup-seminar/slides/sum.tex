\begin{frame}

\frametitle{Truth about Sum}

\begin{itemize}

\item There is more than one way.

\item Magnitude matters.

\item Sign (order) matters.

\end{itemize}

\vspace{\fill}

\begin{center}

Perform sum as if with infinite precision, \\ rounded to fit the desired
resulting precision.

\end{center}

\vspace{\fill}

\footnotesize

\begin{thebibliography}{9}

\bibitem{higham-1983} Nicholas J. Higham. \emph{The Accuracy of Floating Point
Summation}. SIAM J. Sci. Comput., 14(4): 783--799, 1993.

\bibitem{kadric-et-al-2013} E. Kadric, P. Gurniak and A. Dehon. \emph{Accurate
Parallel Floating-Point Accumulation}. Computer Arithmetic (ARITH), 2013, 21st
IEEE Symposium on, pp. 153--162.

\end{thebibliography}

\end{frame}

%\begin{frame}

%Computing the sum of a non-empty set of numbers:

%\begin{codebox}
%\Procname{$\proc{Sum}\p{S}$}
%\li \While $\card{S}>1$ \Do
%\li   $x\gets \proc{Pop}(S)$
%\li   $y\gets \proc{Pop}(S)$
%\li   $\proc{Push}(S, \circ\p{x+y})$
%\End
%\li \Return $\proc{Pop}(S)$
%\end{codebox}

%\begin{itemize}

%\item Cardinality of $S$ decreases in every iteration.

%\item $n-1$ additions, where $n$ is the initial cardinality of $S$.

%\end{itemize}

%\end{frame}
