% Palatino font
\usepackage[T1]{fontenc}
\usepackage{mathpazo}

\setbeamertemplate{navigation symbols}{} % strip navigation symbols
\usefonttheme{serif} % use the sans-serif fonts (Palatino)
\setbeamerfont{title}{series=\bfseries,parent=structure}
\setbeamerfont{frametitle}{series=\bfseries,parent=structure}

\usepackage{textpos}
\usepackage{paralist}
%\usepackage{enumitem}
\usepackage{expdlist}
\usepackage{subfigure}
\usepackage{amssymb, amsmath, amsthm}
\everymath{\displaystyle}

\usepackage[T1]{fontenc}
\usepackage{mathpazo}

\definecolor{shade}{RGB}{250,250,250}
\definecolor{blue}{RGB}{0,76,153}
\definecolor{lightblue}{RGB}{200,200,255}

\newcommand{\code}[1]{\texttt{#1}}
\newcommand{\mono}[1]{\texttt{#1}}
\newcommand{\textt}[1]{\ensuremath{\text{\mono{#1}}}}
\newcommand{\mathmono}[1]{\ensuremath{\text{\mono{#1}}}}
\newcommand{\nonterm}[1]{\ensuremath{\text{\mono{<#1>}}}}
\newcommand{\term}[1]{\ensuremath{\text{\mono{`#1'}}}}
\newcommand{\any}[0]{\ensuremath{\left\langle\bigtriangleup\right\rangle}}
\newcommand{\D}{$\Delta$}

\newcommand{\makeTable}[6][htbp!]
{
	\begin{table}[#1]
	\centering
	\begin{tabular}{#4}
	#5\\
	#6\\
	\end{tabular}
	\end{table}
}

\newcommand{\includeFigure}[4][htbp!]
{
	\begin{figure}[#4]
	\centering
	\IfDecimal{#1}
	{
		\includegraphics[scale=#1]{figures/#2}
	}
	{
		\includegraphics[#1]{figures/#2}
	}
	\caption[]{#3}
	\label{figure:#2}
	\end{figure}
}

\usepackage{clrscode3e}
\usepackage{verbatim}
\usepackage{listings}
\lstset
{
	tabsize=2,
	numbers=left,
	breaklines=true,
%	foregroundcolor=\color{shade},
	framexleftmargin=0.05in,
	basicstyle=\ttfamily\footnotesize,
	numberstyle=\tiny,
%  keywordstyle=\color{green},
%	stringstyle=\color{red},
%	commentstyle=\color{ForestGreen},
  keywords={input},
  mathescape=true,
  captionpos=b,
  columns=fullflexible,
%  escapeinside={(*@}{@*)},
  mathescape=false,
  language=bash
}

\setbeamercolor{title}{fg=blue,bg=shade}
\setbeamercolor{frametitle}{fg=blue,bg=shade}
\setbeamercolor{normal text}{fg=blue,bg=shade}
\setbeamercolor{background canvas}{bg=shade}
\setbeamercolor{normal text}{fg=blue,bg=shade}

\setbeamercolor{bibliography item}{fg=blue}
\setbeamercolor{bibliography entry author}{fg=blue}
\setbeamertemplate{bibliography item}[text]

\usepackage{fancyvrb}

\newcommand{\question}[1]%
{%
  \begin{center}%
  \Large \textbf{#1}%
  \end{center}%
}

\newcommand{\answer}[1]%
{%
  \begin{flushright}%
  --- #1%
  \end{flushright}%
}

\usepackage{ulem}
