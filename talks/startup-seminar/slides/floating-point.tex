\begin{frame}

\frametitle{Floating-Point}

\begin{center}

A floating-point value is a triplet $\p{s,f,e}$, representing the mathematical
value\footnote{Typically, $\beta = 2$.}

$$(-1)^s \cdot f \cdot \beta^e.$$

\end{center}

\vspace{\fill}

Fitting this triplet onto the conventional digital quanta of \\ 8, 16, 32, 64,
etc. bits presents a number of challenges:

\begin{itemize}

\item Should we prioritze $f$ (significand) or $e$ (exponent)?

\item Can we have an arithmetic closed under basic operations?

\item What about exceptional behaviour (e.g., divide by $0$)?

\end{itemize}

\end{frame}


\begin{frame}[fragile]

\frametitle{IEEE 754 Floating-Point}

For the functional programmer:

\begin{lstlisting}[language=haskell,numbers=none]
data FP
  = Normal    Sign Significand Exponent
  | Infinity  Sign
  | NaN       Payload -- Payload carries the reason for the NaN.
  | Zero      Sign
  | Subnormal Sign Significand -- Allows gradual underflow.
\end{lstlisting}

\vspace{\fill}

\begin{itemize}

\item Designed for portability (but doesn't always succeed).

\item Closure is achieved through rounding (next slide).

\end{itemize}

\end{frame}


\begin{frame}[fragile]

\frametitle{IEEE 754 Floating-Point --- Rounding}

\vspace{\fill}

\begin{center}

Perform every operation as if with infinite precision, \\ rounded to fit
the desired resulting precision.

\end{center}

\vspace{\fill}

\footnotesize

\begin{itemize}

\item Round towards positive infinity.

\item Round towards negative infinity.

\item Round towards zero.

\item Round to nearest, ties to even (most common, default).

\item Round to nearest, ties away from zero (optional for base 2).

\item Custom rounding modes are allowed\ldots

\end{itemize}

\vspace{\fill}

\end{frame}
